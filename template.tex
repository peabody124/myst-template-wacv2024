% WACV 2024 Paper Template
% based on the CVPR2023 template (https://media.icml.cc/Conferences/CVPR2023/cvpr2023-author_kit-v1_1-1.zip) with 2-track changes from the WACV 2023 template (https://github.com/wacv-pcs/WACV-2023-Author-Kit)
% based on the CVPR template provided by Ming-Ming Cheng (https://github.com/MCG-NKU/CVPR_Template)
% modified and extended by Stefan Roth (stefan.roth@NOSPAMtu-darmstadt.de)

\documentclass[10pt,twocolumn,letterpaper]{article}

%%%%%%%%% PAPER TYPE  - PLEASE UPDATE FOR FINAL VERSION
%\usepackage[review,algorithms]{wacv}      % To produce the REVIEW version for the algorithms track
\usepackage[review,applications]{wacv}      % To produce the REVIEW version for the applications track
%\usepackage{wacv}              % To produce the CAMERA-READY version
%\usepackage[pagenumbers]{wacv} % To force page numbers, e.g. for an arXiv version

%%%%%%%%%%%%%%%  Packages   %%%%%%%%%%%%%%%
[-IMPORTS-]

% Include other packages here, before hyperref.
\usepackage{graphicx}
\usepackage{amsmath}
\usepackage{amssymb}
\usepackage{booktabs}
\usepackage[numbers]{natbib}

\renewcommand{\citep}[1]{\cite{#1}}
\renewcommand{\citet}[1]{\cite{#1}}

% It is strongly recommended to use hyperref, especially for the review version.
% hyperref with option pagebackref eases the reviewers' job.
% Please disable hyperref *only* if you encounter grave issues, e.g. with the
% file validation for the camera-ready version.
%
% If you comment hyperref and then uncomment it, you should delete
% ReviewTempalte.aux before re-running LaTeX.
% (Or just hit 'q' on the first LaTeX run, let it finish, and you
%  should be clear).
\usepackage[pagebackref,breaklinks,colorlinks]{hyperref}


% Support for easy cross-referencing
\usepackage[capitalize]{cleveref}
\crefname{section}{Sec.}{Secs.}
\Crefname{section}{Section}{Sections}
\Crefname{table}{Table}{Tables}
\crefname{table}{Tab.}{Tabs.}


%%%%%%%%% PAPER ID  - PLEASE UPDATE
\def\wacvPaperID{*****} % *** Enter the WACV Paper ID here
\def\confName{WACV}
\def\confYear{2024}


\begin{document}

%%%%%%%%%%%%%%%%   Title   %%%%%%%%%%%%%%%%
\title{[-doc.title-]}


%%%%%%%%%%%%%%%  Author list  %%%%%%%%%%%%%%%

% \author{
% [# for author in doc.authors #]
% 	[--author.name-]
% 	\inst{[--author.affiliations|join(', ', 'index')--]}
% 	[#- if author.orcid -#]
% 	%\orcidID{[-author.orcid-]}
% 	\href{https://orcid.org/[-author.orcid-]}{\orcidicon}
% 	[#-if author.email-#]
% 	\href{[-author.email-]}{\Letter}
% 	[#-endif-#]	
% 	[#- endif -#]
% 	[#- if not loop.last #] \and [# endif #]
% [# endfor #]
% }

% \institute{
% [# for affiliation in doc.affiliations #]
% {[-affiliation.value-]}
% [#- if not loop.last #] \and [# endif #]
% [# endfor #]
% }

% % if the length of doc.authors is > 2 use author running with just first author et al
% % TODO: need to also parse the name and only show the last word
% [# if doc.authors.length > 2 #]
% \authorrunning{[-doc.authors[0].name-] et al.}
% [# else #]
% [# endif #]

\maketitle

[# if parts.abstract #]
\begin{abstract}
[-parts.abstract-]\\

% [# if doc.keywords #]
% \keywords{[-doc.keywords|join(", ")-]}
% [# endif #]

\end{abstract}
[# endif #]


%%%%%%%%%%%%%%%  Main text   %%%%%%%%%%%%%%%

[-CONTENT-]

[# if parts.acknowledgments #]
%%%%%%%%%%%%% Acknowledgements %%%%%%%%%%%%%
\section*{Acknowledgements}
\footnotesize
[- parts.acknowledgments -]
\normalsize
[# endif #]

[# if options.link #]
\section*{Original article}
\footnotesize
This article is available online at the following URL: \href{[-options.link-]}{[-options.link-]}
\normalsize
[# endif #]

%%%%%%%%%%%%%%   Bibliography   %%%%%%%%%%%%%%
[# if doc.bibliography #]
{\small
\bibliographystyle{ieee_fullname}
\bibliography{[- doc.bibliography | join(", ") -]}
}
[# endif #]


%%%%%%%%%%%%%%   Appendix   %%%%%%%%%%%%%%

%% MyST Note: Appendix isn't a part of the original template but is
%% useful.
[# if parts.appendix #]
\clearpage

%%\section*{Appendix}
[-parts.appendix-]
[# endif #]

\end{document}
